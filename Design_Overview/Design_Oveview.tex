%
%
\documentclass[10pt]{article}
%\usepackage[latin1]{inputenc}
\usepackage{amsmath}
%\usepackage{amsfonts}
\usepackage{amssymb}
\usepackage{graphicx}
\usepackage{hyperref}
\usepackage{vmargin}



%%%%%%%%%%%%%%%%%%%%%%%%%%%%%%
\title{Input Optics Design Overview}
\author{Chris Mueller}
%\ligodraft
%%%%%%%%%%%%%%%%%%%%%%%%%%%%%%%%%%%%%%%%%%%%%%%%%%%%%%%%%%%%%%%%%%%%%
\begin{document}


%=================================================================================================
\section{Input Optics Design Overview}

The previous section \textbf{cite previous section} describes how the requirements placed on the 
input optics by the gravitational wave sensitivity are set.  
It does not; however, explain the difficulties faced in meeting those requirements.  

The largest difficulty in meeting the design requirements is that the input optics 
must be able to operate at input laser powers between 100 mW and 150 W, 
a dynamic range of $10^3$.  
The chief difficulty of operating at such high laser powers is with thermal lensing 
effects in the various optical components of the input optics.  
Thermal lensing is due to the variation of the index of refraction of an optical 
material with temperature.  
When a high power beam passes through a material with non-negligible absorption 
the temperature of the material increases in a Gaussian pattern centered on the beam.  
This Gaussian temperature distribution causes the optical material to act like a power 
dependent lens, making it difficult to maintain mode matching and alignment as the power 
is increased.  
This effect is exaggerated in the input optics because the materials chosen for electro-optic 
modulation and Faraday isolation are chosen for traits other than their low optical absorption.

%--------------------------------------------------------------------------------------------------
\subsection{Differences from Initial LIGO}

%--------------------------------------------------------------------------------------------------
\subsection{Interface with Other Subsystems}


\end{document}
