%
%
\documentclass[10pt]{article}
%\usepackage[latin1]{inputenc}
\usepackage{amsmath}
%\usepackage{amsfonts}
\usepackage{amssymb}
\usepackage{graphicx}
\usepackage{hyperref}
\usepackage{vmargin}



%%%%%%%%%%%%%%%%%%%%%%%%%%%%%%
\title{Input Optics Design Overview}
\author{Chris Mueller}
%\ligodraft
%%%%%%%%%%%%%%%%%%%%%%%%%%%%%%%%%%%%%%%%%%%%%%%%%%%%%%%%%%%%%%%%%%%%%
\begin{document}


%=================================================================================================
\section{Input Optics Design Overview}

The previous section \textbf{cite previous section} describes how the requirements placed on the 
input optics by the gravitational wave sensitivity are set.  
It does not; however, explain the difficulties faced in meeting those requirements.  

The largest difficulty in meeting the design requirements is that the input optics 
must be able to operate at input laser powers between 100 mW and 150 W, 
a dynamic range of $10^3$.  
The chief difficulty of operating at such high laser powers is with thermal lensing 
effects in the various optical components of the input optics.  
Thermal lensing is due to the variation of the index of refraction of an optical 
material with temperature.  
When a high power beam passes through a material with non-negligible absorption 
the temperature of the material increases in a Gaussian pattern centered on the beam.  
This Gaussian temperature distribution causes the optical material to act like a power 
dependent lens, making it difficult to maintain mode matching and alignment as the power 
is increased.  
This effect is exaggerated in the input optics because the materials chosen for electro-optic 
modulation and Faraday isolation must be chosen for traits other than low optical absorption.

Another challenge arises in choosing the coefficient of finesse of the input mode cleaner.  
A triangular cavity's isolation from higher order modes, rejection of the incorrect polarization, 
and its ability to quiet frequency noise all roughly scale with the finesse of the cavity.  
However; higher finesse means higher circulating power in the cavity which leads to thermal 
distortions and the possibility of damage.  

Finally, the seismic noise at both sites is 8-10 orders of magnitude too high to reach 
the required level of frequency noise suppression.  
Reaching this level requires many levels of carefully designed seismic isolation.  
The seismic isolation subsystems are not part of the task of the input optics, but 
the complicated active and passive isolation systems play an important role in 
the design choices of the input optics.  

%--------------------------------------------------------------------------------------------------
\subsection{Differences from Initial LIGO}

The Initial LIGO interferometers were built on largely proven technologies with the goal of 
quickly getting into the business of collecting gravitational wave data\cite{Fritschel2003}.  
The Advanced LIGO era builds on the experience of this time by pushing the state of the art 
farther in order to build a much more sensitive interferometer.  
More than 95\% of the input optics hardware is brand new for the Advanced LIGO era.  

The electro-optic modulator uses a rubidium titanyl phosphate (RTP) crystal instead 
of lithium niobate (LiNbO).  
This material was chosen because the damage threshold is at least a factor of two higher, 
the absorption is a factor of 100 lower at 1064 nm, and the change in the index of 
refraction with temperature (dn/dT) is at least a factor of three lower.  
All of this together gives an electo-optic material with significantly lower thermal 
lensing than its iLIGO counterpart.  
All of this while only taking a $\sim20\%$ reduction in modulation depth per Volt.

In addition the aLIGO EOM employs uses three seperate electrodes on a single crystal 
to simplify resonant circuit design and adjustment and keep the number of scattering 
surfaces to a minimum.
It also has wedged faces to prohibit polarization rotation inside of the crystal which 
gets converted to amplitude modulation at the next polarizer.  





Thermal effects in the input optics were carefully studied during the design phase leading up 
to Advanced LIGO and during the sixth LIGO science run known as Enhanced LIGO\cite{Dooley2012}.  



%--------------------------------------------------------------------------------------------------
\subsection{Interface with Other Subsystems}



\begin{thebibliography}{9}
	\bibitem{Dooley2012} Dooley, Katherine L., Muzammil A. Arain, David Feldbaum, Valery V. Frolov, 
		Matthew Heintze, Daniel Haok, Efim A. Khazanov, Antonio Lucianetti, Rodica M. Matrin, Guido Mueller, 
		Oleg Palashov, Volker Quetschke, David H. Reitze, R.L. Savage, D.B. Tanner, Luke F. Williams, 
		and Wan Wu.  ``Characterization of thermal effects in the Enhance LIGO Input Optics.'' 
		Review of Scientific Instruments 83, 033109-1/12, 2012.
	\bibitem{Fritschel2003}
		Fritschel, Peter ``Second generation instruments for the Laser Interferometer 
		Gravitational Wave Observatory (LIGO)'', Proc. SPIE 4856, Gravitational-Wave Detection, 
		282 (February 21, 2003); doi:10.1117/12.459090
\end{thebibliography}	

\end{document}


